\documentclass[conference, 11pt]{IEEEtran} 
\usepackage{verbatim}
\usepackage{multirow} \usepackage{enumerate}
\usepackage{amsmath,enumerate} \usepackage{amsthm}
\usepackage{algorithm}
\usepackage{algorithmic}
\usepackage{pstricks}
\usepackage{amssymb, latexsym}
\usepackage{graphicx}
\DeclareGraphicsRule{*}{mps}{*}{}

\usepackage{pgf}
\usepackage{tikz}
\usetikzlibrary{decorations.pathmorphing} % LATEX and plain TEX when using Tik Z
\usetikzlibrary{positioning}
\usetikzlibrary{er}
\tikzstyle{vx}=[draw,circle,fill=black!50,minimum size=2pt, inner sep=1pt, node distance=15mm]
\tikzstyle{bup}=[semithick, decoration={bent, aspect=.3, amplitude=4}, decorate, ->, >=stealth]
\tikzstyle{bdn}=[semithick, decoration={bent, aspect=.3, amplitude=-4}, decorate, ->, >=stealth]
\tikzstyle{BUP}=[thick, decoration={bent, aspect=.3, amplitude=8}, decorate, ->, >=stealth]
\tikzstyle{BDN}=[thick, decoration={bent, aspect=.3, amplitude=-8}, decorate, ->, >=stealth]
\tikzstyle{MUP}=[thick, decoration={bent, aspect=.3, amplitude=16}, decorate, ->, >=stealth]
\tikzstyle{MDN}=[thick, decoration={bent, aspect=.3, amplitude=-16}, decorate, ->, >=stealth]
\tikzstyle{str}=[semithick, decorate, ->, >=stealth]
\tikzstyle{cr}=[draw, circle, fill=black!25,minimum size=150pt]

% \paperheight=11in \paperwidth=8.5in \textheight=9.0in
% \textwidth=6.5in \voffset=-.875in \hoffset=-.875in
\newenvironment{code} {\begin {quote}\begin{footnotesize}}
    {\end{footnotesize}\end{quote}}

% \oddsidemargin 0.0 in \evensidemargin 0.0 in
\newenvironment{enumeratealpha}
{\begin{enumerate}[(a{\textup{)}}]}{\end{enumerate}}

\theoremstyle{definition}
\newtheorem{lem-rule}{Rule}

% text macros
\def\cI{{\mathcal I}} \def\cR{{\mathcal R}} \def\cE{{\mathcal E}}
\def\cC{{\mathcal C}} \def\cF{{\mathcal F}} \def\cU{{\mathcal U}}
\def\cH{{\mathcal H}} \def\cD{{\mathcal D}} \def\cB{{\mathcal B}}
\def\cQ{{\mathcal Q}} \def\cV{{\mathcal V}} \def\cS{{\mathcal S}}
\def\cG{{\mathcal G}} \def\cA{{\mathcal A}}

\def\cId{{$\mathcal I$}} \def\cRd{{$\mathcal R$}} \def\cEd{{$\mathcal
    E$}} \def\cCd{{$\mathcal C$}} \def\cFd{{$\mathcal F$}}
\def\cUd{{$\mathcal U$}} \def\cHd{{$\mathcal H$}} \def\cDd{{$\mathcal
    D$}} \def\cBd{{$\mathcal B$}} \def\cQd{{$\mathcal Q$}}
\def\cVd{{$\mathcal V$}} \def\cSd{{$\mathcal S$}} \def\cGd{{$\mathcal
    G$}} \def\cAd{{$\mathcal A$}}

\bibliographystyle {IEEEtranS}
\begin {document}

\title{IEEE Paper} 

\author{\IEEEauthorblockN{J. Paul Daigle}
\IEEEauthorblockA{Department of Computer Science\\
Georgia State University\\
Atlanta, Georgia 30303\\
Email: jdaigle1@student.gsu.edu}
}

\maketitle

\begin{abstract}

\end{abstract}

\section{Results} 
In this section, we evaluate the performance of our algorithm in two cases. First, we compare our results to a baseline 2-approximation algorithm for unweighted vertex cover.\cite{500824} Next, we evaluate the performance of our algorithm against the distributed combinatorial 2-approximation algorithm for weighted vertex cover presented by Koufogiannakis and Young.\cite{1582746} The simulations were programmed using Python.\footnote{The code for this project is available on google code as open source software, this paper references changeset 6fa22bd4b8d9.} To adapt the framework based target coverage algorithm presented by Dhawan and Prasad\cite{IPDPS.2008.45361}, edges between nodes are represented as targets. 
\subsection{Unweighted Cover}
To calculate the unweighted cover, a graph containing 100 unweighted nodes was assigned a random number of edges in logarithmic steps between 100 and 4500 (forming a clique). 20 graphs were generated in each size. Vertex covers were calculated separately using the framework algorithm versus the standard reduction algorithm. The framework produced covers that were consitent with a constant 2-$\epsilon$ approximation. Figure~\ref{fig:unweight_covers} shows the average performance of both algorithms with the total degree of the tested graphs compared to the size of the vertex cover. Sensors were deployed in a square area and assumed to start with some small variation in battery life. Communication rounds were not modeled. The 2-approximation algorithm is known to have a communications cost of $O(log p)$ and the framework algorithm is bounded by $O(\Delta^\Delta)$, which can be viewed as a constant for sparse graphs.

\begin{figure}[width=2in]
  \label{fig:unweight_covers}
  \caption{Unweighted Case Versus 2-approximation}
  % GNUPLOT: LaTeX picture
\setlength{\unitlength}{0.240900pt}
\ifx\plotpoint\undefined\newsavebox{\plotpoint}\fi
\begin{picture}(900,540)(0,0)
\sbox{\plotpoint}{\rule[-0.200pt]{0.400pt}{0.400pt}}%
\put(191.0,131.0){\rule[-0.200pt]{4.818pt}{0.400pt}}
\put(171,131){\makebox(0,0)[r]{ 25}}
\put(830.0,131.0){\rule[-0.200pt]{4.818pt}{0.400pt}}
\put(191.0,205.0){\rule[-0.200pt]{4.818pt}{0.400pt}}
\put(171,205){\makebox(0,0)[r]{ 30}}
\put(830.0,205.0){\rule[-0.200pt]{4.818pt}{0.400pt}}
\put(191.0,279.0){\rule[-0.200pt]{4.818pt}{0.400pt}}
\put(171,279){\makebox(0,0)[r]{ 35}}
\put(830.0,279.0){\rule[-0.200pt]{4.818pt}{0.400pt}}
\put(191.0,352.0){\rule[-0.200pt]{4.818pt}{0.400pt}}
\put(171,352){\makebox(0,0)[r]{ 40}}
\put(830.0,352.0){\rule[-0.200pt]{4.818pt}{0.400pt}}
\put(191.0,426.0){\rule[-0.200pt]{4.818pt}{0.400pt}}
\put(171,426){\makebox(0,0)[r]{ 45}}
\put(830.0,426.0){\rule[-0.200pt]{4.818pt}{0.400pt}}
\put(191.0,500.0){\rule[-0.200pt]{4.818pt}{0.400pt}}
\put(171,500){\makebox(0,0)[r]{ 50}}
\put(830.0,500.0){\rule[-0.200pt]{4.818pt}{0.400pt}}
\put(191.0,131.0){\rule[-0.200pt]{0.400pt}{4.818pt}}
\put(191,90){\makebox(0,0){50}}
\put(191.0,480.0){\rule[-0.200pt]{0.400pt}{4.818pt}}
\put(214.0,131.0){\rule[-0.200pt]{0.400pt}{4.818pt}}
\put(214,90){\makebox(0,0){51}}
\put(214.0,480.0){\rule[-0.200pt]{0.400pt}{4.818pt}}
\put(236.0,131.0){\rule[-0.200pt]{0.400pt}{4.818pt}}
\put(236,90){\makebox(0,0){52}}
\put(236.0,480.0){\rule[-0.200pt]{0.400pt}{4.818pt}}
\put(259.0,131.0){\rule[-0.200pt]{0.400pt}{4.818pt}}
\put(259,90){\makebox(0,0){53}}
\put(259.0,480.0){\rule[-0.200pt]{0.400pt}{4.818pt}}
\put(282.0,131.0){\rule[-0.200pt]{0.400pt}{4.818pt}}
\put(282,90){\makebox(0,0){54}}
\put(282.0,480.0){\rule[-0.200pt]{0.400pt}{4.818pt}}
\put(305.0,131.0){\rule[-0.200pt]{0.400pt}{4.818pt}}
\put(305,90){\makebox(0,0){55}}
\put(305.0,480.0){\rule[-0.200pt]{0.400pt}{4.818pt}}
\put(327.0,131.0){\rule[-0.200pt]{0.400pt}{4.818pt}}
\put(327,90){\makebox(0,0){97}}
\put(327.0,480.0){\rule[-0.200pt]{0.400pt}{4.818pt}}
\put(350.0,131.0){\rule[-0.200pt]{0.400pt}{4.818pt}}
\put(350,90){\makebox(0,0){98}}
\put(350.0,480.0){\rule[-0.200pt]{0.400pt}{4.818pt}}
\put(373.0,131.0){\rule[-0.200pt]{0.400pt}{4.818pt}}
\put(373,90){\makebox(0,0){140}}
\put(373.0,480.0){\rule[-0.200pt]{0.400pt}{4.818pt}}
\put(396.0,131.0){\rule[-0.200pt]{0.400pt}{4.818pt}}
\put(396,90){\makebox(0,0){141}}
\put(396.0,480.0){\rule[-0.200pt]{0.400pt}{4.818pt}}
\put(418.0,131.0){\rule[-0.200pt]{0.400pt}{4.818pt}}
\put(418,90){\makebox(0,0){183}}
\put(418.0,480.0){\rule[-0.200pt]{0.400pt}{4.818pt}}
\put(441.0,131.0){\rule[-0.200pt]{0.400pt}{4.818pt}}
\put(441,90){\makebox(0,0){225}}
\put(441.0,480.0){\rule[-0.200pt]{0.400pt}{4.818pt}}
\put(464.0,131.0){\rule[-0.200pt]{0.400pt}{4.818pt}}
\put(464,90){\makebox(0,0){226}}
\put(464.0,480.0){\rule[-0.200pt]{0.400pt}{4.818pt}}
\put(486.0,131.0){\rule[-0.200pt]{0.400pt}{4.818pt}}
\put(486,90){\makebox(0,0){268}}
\put(486.0,480.0){\rule[-0.200pt]{0.400pt}{4.818pt}}
\put(509.0,131.0){\rule[-0.200pt]{0.400pt}{4.818pt}}
\put(509,90){\makebox(0,0){310}}
\put(509.0,480.0){\rule[-0.200pt]{0.400pt}{4.818pt}}
\put(532.0,131.0){\rule[-0.200pt]{0.400pt}{4.818pt}}
\put(532,90){\makebox(0,0){352}}
\put(532.0,480.0){\rule[-0.200pt]{0.400pt}{4.818pt}}
\put(555.0,131.0){\rule[-0.200pt]{0.400pt}{4.818pt}}
\put(555,90){\makebox(0,0){394}}
\put(555.0,480.0){\rule[-0.200pt]{0.400pt}{4.818pt}}
\put(577.0,131.0){\rule[-0.200pt]{0.400pt}{4.818pt}}
\put(577,90){\makebox(0,0){436}}
\put(577.0,480.0){\rule[-0.200pt]{0.400pt}{4.818pt}}
\put(600.0,131.0){\rule[-0.200pt]{0.400pt}{4.818pt}}
\put(600,90){\makebox(0,0){478}}
\put(600.0,480.0){\rule[-0.200pt]{0.400pt}{4.818pt}}
\put(623.0,131.0){\rule[-0.200pt]{0.400pt}{4.818pt}}
\put(623,90){\makebox(0,0){561}}
\put(623.0,480.0){\rule[-0.200pt]{0.400pt}{4.818pt}}
\put(645.0,131.0){\rule[-0.200pt]{0.400pt}{4.818pt}}
\put(645,90){\makebox(0,0){603}}
\put(645.0,480.0){\rule[-0.200pt]{0.400pt}{4.818pt}}
\put(668.0,131.0){\rule[-0.200pt]{0.400pt}{4.818pt}}
\put(668,90){\makebox(0,0){645}}
\put(668.0,480.0){\rule[-0.200pt]{0.400pt}{4.818pt}}
\put(691.0,131.0){\rule[-0.200pt]{0.400pt}{4.818pt}}
\put(691,90){\makebox(0,0){728}}
\put(691.0,480.0){\rule[-0.200pt]{0.400pt}{4.818pt}}
\put(714.0,131.0){\rule[-0.200pt]{0.400pt}{4.818pt}}
\put(714,90){\makebox(0,0){770}}
\put(714.0,480.0){\rule[-0.200pt]{0.400pt}{4.818pt}}
\put(736.0,131.0){\rule[-0.200pt]{0.400pt}{4.818pt}}
\put(736,90){\makebox(0,0){853}}
\put(736.0,480.0){\rule[-0.200pt]{0.400pt}{4.818pt}}
\put(759.0,131.0){\rule[-0.200pt]{0.400pt}{4.818pt}}
\put(759,90){\makebox(0,0){895}}
\put(759.0,480.0){\rule[-0.200pt]{0.400pt}{4.818pt}}
\put(782.0,131.0){\rule[-0.200pt]{0.400pt}{4.818pt}}
\put(782,90){\makebox(0,0){978}}
\put(782.0,480.0){\rule[-0.200pt]{0.400pt}{4.818pt}}
\put(805.0,131.0){\rule[-0.200pt]{0.400pt}{4.818pt}}
\put(805,90){\makebox(0,0){1061}}
\put(805.0,480.0){\rule[-0.200pt]{0.400pt}{4.818pt}}
\put(827.0,131.0){\rule[-0.200pt]{0.400pt}{4.818pt}}
\put(827,90){\makebox(0,0){1144}}
\put(827.0,480.0){\rule[-0.200pt]{0.400pt}{4.818pt}}
\put(850.0,131.0){\rule[-0.200pt]{0.400pt}{4.818pt}}
\put(850,90){\makebox(0,0){1225}}
\put(850.0,480.0){\rule[-0.200pt]{0.400pt}{4.818pt}}
\put(191.0,131.0){\rule[-0.200pt]{0.400pt}{88.892pt}}
\put(191.0,131.0){\rule[-0.200pt]{158.753pt}{0.400pt}}
\put(850.0,131.0){\rule[-0.200pt]{0.400pt}{88.892pt}}
\put(191.0,500.0){\rule[-0.200pt]{158.753pt}{0.400pt}}
\put(70,315){\makebox(0,0){Cover Size}}
\put(520,29){\makebox(0,0){Total Degree}}
\sbox{\plotpoint}{\rule[-0.600pt]{1.200pt}{1.200pt}}%
\put(191,323){\circle*{18}}
\put(214,264){\circle*{18}}
\put(236,205){\circle*{18}}
\put(259,264){\circle*{18}}
\put(282,234){\circle*{18}}
\put(305,293){\circle*{18}}
\put(327,382){\circle*{18}}
\put(350,323){\circle*{18}}
\put(373,382){\circle*{18}}
\put(396,411){\circle*{18}}
\put(418,382){\circle*{18}}
\put(441,441){\circle*{18}}
\put(464,500){\circle*{18}}
\put(486,470){\circle*{18}}
\put(509,470){\circle*{18}}
\put(532,441){\circle*{18}}
\put(555,470){\circle*{18}}
\put(577,470){\circle*{18}}
\put(600,441){\circle*{18}}
\put(623,470){\circle*{18}}
\put(645,470){\circle*{18}}
\put(668,500){\circle*{18}}
\put(691,500){\circle*{18}}
\put(714,500){\circle*{18}}
\put(736,500){\circle*{18}}
\put(759,470){\circle*{18}}
\put(782,470){\circle*{18}}
\put(805,470){\circle*{18}}
\put(827,500){\circle*{18}}
\put(850,500){\circle*{18}}
\sbox{\plotpoint}{\rule[-0.400pt]{0.800pt}{0.800pt}}%
\put(191,161){\raisebox{-.8pt}{\makebox(0,0){$\Box$}}}
\put(214,249){\raisebox{-.8pt}{\makebox(0,0){$\Box$}}}
\put(236,146){\raisebox{-.8pt}{\makebox(0,0){$\Box$}}}
\put(259,190){\raisebox{-.8pt}{\makebox(0,0){$\Box$}}}
\put(282,205){\raisebox{-.8pt}{\makebox(0,0){$\Box$}}}
\put(305,220){\raisebox{-.8pt}{\makebox(0,0){$\Box$}}}
\put(327,323){\raisebox{-.8pt}{\makebox(0,0){$\Box$}}}
\put(350,293){\raisebox{-.8pt}{\makebox(0,0){$\Box$}}}
\put(373,382){\raisebox{-.8pt}{\makebox(0,0){$\Box$}}}
\put(396,352){\raisebox{-.8pt}{\makebox(0,0){$\Box$}}}
\put(418,367){\raisebox{-.8pt}{\makebox(0,0){$\Box$}}}
\put(441,441){\raisebox{-.8pt}{\makebox(0,0){$\Box$}}}
\put(464,426){\raisebox{-.8pt}{\makebox(0,0){$\Box$}}}
\put(486,470){\raisebox{-.8pt}{\makebox(0,0){$\Box$}}}
\put(509,441){\raisebox{-.8pt}{\makebox(0,0){$\Box$}}}
\put(532,441){\raisebox{-.8pt}{\makebox(0,0){$\Box$}}}
\put(555,470){\raisebox{-.8pt}{\makebox(0,0){$\Box$}}}
\put(577,470){\raisebox{-.8pt}{\makebox(0,0){$\Box$}}}
\put(600,470){\raisebox{-.8pt}{\makebox(0,0){$\Box$}}}
\put(623,456){\raisebox{-.8pt}{\makebox(0,0){$\Box$}}}
\put(645,470){\raisebox{-.8pt}{\makebox(0,0){$\Box$}}}
\put(668,456){\raisebox{-.8pt}{\makebox(0,0){$\Box$}}}
\put(691,470){\raisebox{-.8pt}{\makebox(0,0){$\Box$}}}
\put(714,470){\raisebox{-.8pt}{\makebox(0,0){$\Box$}}}
\put(736,485){\raisebox{-.8pt}{\makebox(0,0){$\Box$}}}
\put(759,470){\raisebox{-.8pt}{\makebox(0,0){$\Box$}}}
\put(782,485){\raisebox{-.8pt}{\makebox(0,0){$\Box$}}}
\put(805,485){\raisebox{-.8pt}{\makebox(0,0){$\Box$}}}
\put(827,485){\raisebox{-.8pt}{\makebox(0,0){$\Box$}}}
\put(850,485){\raisebox{-.8pt}{\makebox(0,0){$\Box$}}}
\sbox{\plotpoint}{\rule[-0.200pt]{0.400pt}{0.400pt}}%
\put(191.0,131.0){\rule[-0.200pt]{0.400pt}{88.892pt}}
\put(191.0,131.0){\rule[-0.200pt]{158.753pt}{0.400pt}}
\put(850.0,131.0){\rule[-0.200pt]{0.400pt}{88.892pt}}
\put(191.0,500.0){\rule[-0.200pt]{158.753pt}{0.400pt}}
\end{picture}

\end{figure}	

\bibliography{vertex_bib}
\end {document}
